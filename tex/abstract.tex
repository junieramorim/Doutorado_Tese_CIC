Command and Control(C2) is historically linked with the classical military strategies with a central commander and an inflexible hierarchical chain among the elements or teams. Overall, the main objective of C2 is an optimized resource application to accomplish a mission. However, in a modern C2 context, dynamism of mission, team, and environment is a necessary assumption and thus the team organization to accomplish a mission becomes a challenge. Furthermore, the C2 strategy choice, represented by the level of information spread and the decision making, is not well explored in such dynamism by state-of-the art research. To address these issues, our objective is to apply an approach using Self-Adaptive Systems (SAS) concepts implemented as Dynamic Software Product Lines (DSPL) to represent elements being organized as a team. The most suitable C2 strategy applied to this team permits efficient coordination among and configuration of its elements for goal achievement. Under context changes, either configuration of team members or even their coordination can change accordingly to maintain the mission accomplishment. We envision a preliminary assessment by simulation in drone surveillance operations.


(Final para análise)Given the complexity in dynamic scenarios, the interaction and coordination among DSPL are not currently well defined and explored by the literature. This product lines coordination, proposed by the present study, can provide a model to permit an optimized C2 strategy adoption to define the interaction among elements of a team represented by autonomous systems.
