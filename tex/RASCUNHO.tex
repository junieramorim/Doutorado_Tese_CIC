
The future possibilities that describes all circumstances that an entity can be, defines its endeavour space. Monitoring is an important resource to provide entities the capability of to act with agility and be able to be applied in different portions of the endeavour space within the acceptable quality level. 



One justification: 
We can find many works related to C2 context that applies an agent modelling to represent it and, in some cases, simulate it. However, the complexity that naturally exists in this kind of systems, becomes this modelling strategy not reliable due to the fact that it is not possible to validate it.

This limitation occurs due to the evolved system complexity and some characteristics of the agent based systems. (insert reference of the article: Evaluating the agility of adaptive command and control networks from a cyber complex adaptive systems perspective)


=======


Entretanto, o problema identificado pode ser estendido para outros cenários dotados de sistemas que exijam dinamismo e adaptabilidade a mudanças de contexto. Problemas semelhantes tem sido abordados como Sistemas Multi Agentes (SMA), mas com uma complexidade maior, se considerarmos que SMA seria uma evolução da DSPL, possuidora de uma estrutura mais complexa (artigo SVEN).



As mudanças, tanto nas configurações dos elementos individuais, quanto do conjunto de elementos que compõem o time, são motivadas pelas mudanças de contexto. O contexto é definido pelo ambiente onde o qual o agente encontra-se inserido, bem como seu próprio status. 



Dessa maneira, a presente pesquisa busca introduzir o conceito novo de Múltiplas Linhas de
Produtos de Software Dinâmicas (MLPSD) estendendo sua aplicação na modelagem de soluções
que configuram a formação de times cujos elementos são componentes de software que se
reconfiguram para adaptarem-se às mudanças de contexto, atendendo aos objetivos de qualidade
exigidos e realizando as tarefas determinadas.
Tal abordagem constituirá uma nova estratégia de compreensão e solução de problemas em
diversos domínios que utilizam os conceitos de sistemas dinâmicos, considerando suas
reconfigurações guiadas por objetivos de qualidade (custo, qualidade, tempo de resposta, etc),
onde seus elementos colaboram entre si na execução de atividades. A avaliação do presente
estudo será feita por meio do uso de sistemas de simulação, onde poderemos observar como será
o comportamento desse conjunto de LPSD e suas interações.


Vários modelos utilizados em DSPL foram propostos e validados. O presente estudo pretende realizar uma revisão sistemática e detalhada dessa literatura para caracterizar a evolução em DSPL com foco na qualidade (QDSPL), selecionando um modelo que caracterize essa evolução. O modelo escolhido deverá ser capaz de manter o foco no objetivo, mapeando features e contextos dinâmicos dando suporte à inserção e remoção de capacidades.
Os estudos serão aplicados e validados no domínio de Comando e Controle, em sistemas de simulação de C2 que utilizam SMA. Esses sistemas multiagentes representam uma forma de implementação de DSPL, incorporando a preocupação com a qualidade e estabilidade das tarefas executadas pelo sistema.
Nesse domínio, os modelos serão implementados de modo a colher resultados empíricos por meio de simulações, que evidenciem a teoria tratada.


<<<< ESCREVER NA INTRODUCAO UMA BREVE HISTORIA DO C2 NA ERA INDUSTRIAL E DEPOIS NA ERA DA INFORMAÇÀO, DESTACANDO ASPECTOS IMPORTANTES DESSA MIGRACAO E SEUS IMPACTOS NA AGILITY (C2)>>>>






