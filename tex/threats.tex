%Threats to validity
% scenario more dynamic
% other variables change

Basically, the methodology applied in this study was action research to explore weaknesses and not answered questions related to C2 application and its challenges. The SPL approach applied to prospect the solution proposed required some simplification in the original problem. Thus, for this process, the following threats to validity and corresponding mitigation strategies are presented in the following: 

\begin{itemize}
   \item \textbf{Conclusion validity}: 
   
   \item \textbf{Internal validity}: 
   
   \item \textbf{Construct validity}: To assure that the network structures chosen were suitable to simplify NEC approach applied to the five maturity levels in C2 Approach Space, all connectivity characteristics of each C2 approach were analyzed and compared with each network topology used as approximation. Additionally, mixed structures with complex nodes connections were ignored to becomes the solution feasible.
   
   
   \item \textbf{External validity}: This work considers a dynamic scenario to analyze problem and solution. This dynamism involves changes in the context, i.e., changes in the environment, in the element itself, and in the mission. However, our scope did not consider the environment in a broader sense. Since the complexity related to this dimension, we are working only on aspects that can be perceived by the sensors, e.g., weather changes that a humidity sensor was capable to perceive.
\end{itemize}