\section{Command and Control}

Under DoD definition, C2 is linked to the exercise of authority and direction by properly designated commander over assigned and attached forces in the accomplishment of the mission \cite{}. This definition can bring confusion and is limited in terms to research aspects and does not evolves many other C2 applications in all possible contexts. NATO extended this definition to the functions of commanders, staffs, and other Command and Control bodies in maintaining the combat readiness of their forces, preparing operations, and directing troops in the performance of their tasks \cite{FRANCE2014}.

Empower the individuals on the edge of the organization, spreading the information in the right way and compatible quantity, is the new challenge to support a wider C2 definition. \textit{Alberts and Hayes} in \cite{Power01} presented the term \textit{Power to the Edge} aiming to expose the idea of take the information to all elements in the organization. 

\textit{Alberts and Hayes} in \cite{Power01} presented the key dimensions C2: Allocation of Decision Rights, Patterns of Interaction and Distribution of Information. These dimensions compose the C2 Approach Space described by \textit{Alberts et al.} in \cite{Alberts2006} with its archetypes oriented to Network-Enabled Capabilities (NEC). Each C2 strategy adopted is related to a certain maturity level. Table \ref{table:archetypes1} lists the main characteristics of each \gls{c2strategy}.

\input{tex/table01.tex}

\textbf{a) Allocation of Decison Rights (ADR)}

The aiming is unity of purpose. Minimal distribution of Decision Rights.
Leadership changes in Information Age.

It belongs to the individuals or organizations accepted as authoritative sources on the choices related to a particular topic under some specific circumstance or situation.




\textbf{b) Patterns of Interaction (PoI)}

PoI defines the network used by the entities. According to \textit{Alberts and Hayes} in~\cite{Alberts2006}, it is possible to identify four different types of networks, not related to hierarchy, that makes an approach of C2 in Information age militaries.

\begin{itemize}
    \item Fully connected: every entity is connected to every other, or there is an interaction among all of them;
    \item Random networks: each entity has the same probability of interacting with any other;
    \item Scale-free networks: a few entities have a very large number of connections or interactions with other entities;
    \item Small world networks: very high cluster coefficient, with a number of connections around $log~N$, where $N$ is the total of nodes \cite{small01}.
\end{itemize}

According to \cite{Alberts2006}, the richest network structure should be the one with a scale-free in high level, the intermediate level composed by small world networks and at local level, the using of fully connected network structure.(page 107)

\textit{Zughe and Sun} \cite{small02} proposed a solution to represent a Small-World network using a virtual ring topology. In that work, it is presented some aspects that permits a simpler approach to represent this kind off communication strucutre.

NATO in SAS-060 \cite{nato01} recommended following transition rules in case of endeavour complexity  changes showed in Table \ref{table:table02}.

\input{tex/table02.tex}



\textbf{c) Distribution of Information (DoI)}

How the information is shared, following the Information Exchange Requirements (IRE) and get the shared awareness


Based on this, it is presented a conceptual model to C2 Agility, that is the capability of C2 to successfully effect, cope with or exploit changes in circumstances \cite{ABAR201713} 

The term Network Enabled Capability (NEC) looks for achieving the effect with the best usage of information systems. It was applied in C2 approach and defined five network-enabled approaches showed in Figure \refFig{nec}: 

\begin{itemize}
    \item Edge C2
    \item Collaborative C2
    \item coordinated C2
    \item De-Conflicted C2
    \item Conflicted C2
\end{itemize}

These approaches are selected according to the mission and a set of circumstances. Different \gls{context} requires specific C2 approaches. When some of these elements change, it is possible that some C2 approaching is not suitable anymore. The set of possible approaches available to be employed defines the Endeavour Space~\cite{FRANCE2014}.

Communications improvements conducts the entities to a C2 Approach closer to the edge. This networked structure becomes easier the awareness sharing due to the information sharing and collaboration among entities.

\input{tex/table03.tex}

The network structure and robustness facilitates to share awareness among dispersed entities. When the Network Enabled C2 is more structured, it makes feasible to move to the edge. 

\figura[!h]{GRAPH01}{NEC Approaches to C2}{nec}{width=0.7\textwidth}%

In C2, the agility is linked with agile forces and operational concepts. The six dimensions of agility, that combines agile individuals, organizations and C2 Systems, are listed bellow~\cite{Power01}:

\begin{itemize}
    \item \textbf{Robustness:}it is the ability to maintain the effectiveness across a range of tasks, conditions and situations;
    \item \textbf{Resilience:} it is the ability to recover or adjust from a damage or perturbation in the environment;
    \item \textbf{Responsiveness:} it is the ability to react to a change in the environment in a suitable interval of time;
    \item \textbf{Flexibility:} it is the ability to employ multiple ways to succeed and the capacity to switch between these ways;
    \item \textbf{Innovation:} it is the ability to do new things or in different ways;
    \item \textbf{Adaptation:} it is the ability to change the organization or the processes.
\end{itemize}

These dimensions form the agility enablers. In this scenario, agility is a dependent variable that changes with variations of any of these variables (Responsiveness, Versatility, Flexibility, Resilience, Innovativeness, Adaptability) \cite{FRANCE2014}.

According to the NATO Reports~\cite{FRANCE2014}, \textbf{agility} is the capability to successfully effect, cope with and/or exploit changes in circumstances. Change in circumstances are defined as change in the state of other entities, in the environment, in  itself state or any combination of these three options.


C2 Agility relates the entities available and the mission challenges with a correct C2 approach for each situation. When the circumstances are dynamic, this agility is required to keep the mission execution. The changes, dynamically, in the C2 approaches to another one more appropriate defines the maneuver in C2 space.

This ability to maneuver in the C2 Approach Space involves the recognition of the significance of changes in circumstances that affect the appropriateness of one’s C2 Approach, the comprehension of which C2 Approach is more appropriate to a new particular circumstance, and to be able to change to a more appropriate C2 approach.




===> Detect - decide - act - desired effect  (scale graph in time to measure the agility existence or the agility lack) (page 64/65 in \cite{FRANCE2014})

When all possible changes in circumstances that may cause impacts in an entity, and these possibilities are systematically identified and mapped, they create the Endeavour Space ~\cite{FRANCE2014}.

We can measure the C2 quality through the mission accomplishment level, since C2 is not the end by itself. It is an important tool that contributes to an efficient resources deployment to achieve an aim.


====> Rewrite the following lines from NATO Report \cite{FRANCE2014}

The first, and perhaps most important, task for anyone responsible for a function is to decide how the function should be accomplished. This specifies the specific tasks to be undertaken and creates the conditions that shape the behaviours that emerge as various tasks are being carried out in the context of an Endeavour.
C2 is no exception to this rule. If there was one approach to C2 that worked well for all missions and
circumstances, then there would be no reason to revisit how C2 should be approached. If appropriate choices were always made and circumstances did not change then there would be no need to revisit the selection of an approach to C2 on a continuing basis. However, there is evidence that there is no ‘one size fits all’ approach to C2. There is also ample evidence that inappropriate choices are made as well as ample evidence that circumstances change in ways that affect the quality of C2.
Furthermore,

<==========

The C2 agility can be improved if the agility to switch between different C2 Approaches is increased. The endeavour complexity is related to the C2 Approach adopted (see fig 4.4 pag 78 in \cite{FRANCE2014})

With this concept of C2 Approach, NATO Report~\cite{FRANCE2014} defined C2 Agility as a function of C2 Approach Agility and C2 Maneuver Agility. To guarantee C2 Agility, it is necessary to have situation awareness and self-monitoring. These two elements will permit C2 Approach changes according to the circumstances.




>>>>>> Desenvolver as hipóteses do nosso trabalho (mudança de estratégias de C2. Como mudar? Qual é a melhor? Como medir?? Como simular? etc)




\subsection {Functions of C2}

The following functions, applicable to many endeavors (military, civilian and industrial), are related to C2 and describes steps to be fellow:
\begin{itemize}
    \item Establishing intent
    \item Determining roles
    \item Establishing rules and constraints
    \item Monitoring and assessing the situation and the progress
\end{itemize}

However, it is necessary to consider the leadership factor. How the commanders or managers are good as leaders guiding the team to the mission or task accomplishment. To this aspect, it is necessary to add the following functions:

\begin{itemize}
    \item Inspiring, engendering trust and motivating
    \item Training and education
    \item provisioning
\end{itemize}

These three last functions are related to human aspects and behaviors. They depend on the actor that are as commander or manager.


\subsection {Network-Enabled Capabilities}

The Network Enabled Capabilities - NCW (England) or Network Centric Warfare (USA) - NEC was created to increase the action power through a shared awareness that can be synchronized using a specific network structure and organization to create entities interaction \cite{Alberts2000}. 

\textit{Alberts et al.} presented in \cite{nato01} the five NATO NEC C2 Maturity levels that represents a number of C2 Approaches to deal with different contexts. Each of these approaches corresponds to a specific region in the C2 Approach Space.




\section{Endeavor Space}

The set of all requirements, circumstances and conditions of a mission forms the endeavor space. \textit{Alberts et al.} in  \cite{Alberts2011} defined agility of an entity as the area inside the endeavor space that it is capable to act according to the requirements.

Each region of the endeavor space requires an option in C2 Approach Space to deal with the challenges and to perform the mission given. Different circumstances and other context changes implies in different regions in endeavor space.

A computational view of endeavor space is a search space to the C2 functions application. At different moments, these functions have a specific region of endeavor space as domain values, that conduct to the result or objective, e.g., mission accomplishment. The area formed by the sum of all regions where an entity operates defines the entity's agility in a general way \cite{Alberts2011}.




\section {C2 Agility}

C2 Agility is a function of C2 approach agility and C2 Maneuver agility and represents the capability of an entity to adapt itself to deal with different circumstances. This adaptation can be done through a fine tune with no C2 Approach change, or performing an appropriate C2 approach selection to improve capabilities.

The C2 approach space, according to the NATO NEC Maturity Model, is divided in five regions as cubes that corresponds to specific C2 approaches that can be adopted. However, the selection inside the C2 Approach space can be done within each of these cubes doing a fine adjustment of the C2 Approach to deal with a new circumstance. This adaptation can be expressed as an entity reconfiguration or tasks reallocation.

All entities have a set of possible C2 Approach that can be adopted during context changes to satisfy new requirements. This capacity is expressed as the ability to recognize the context changes and the incompatibility of the C2 Approach applied, the identification of an adequate C2 Approach according to the new circumstance, and the change to this new C2 Approach selected within the time slice defined.

Equation \eqref{c2a1} shows the C2 agility metric representation through a fraction of the endeavor space where the entities can act successfully \cite{Alberts2017}. The endeavor space portions \textit{S1, S2} and \textit{S3} are three regions inside the endeavor space that the entities can act according to some quality parameters, e.g., effectiveness and timeliness. 

\begin{equation}
\label{c2a1}
 C2_{metric} = \frac{S1 + S2 + S3}{endeavor\ space}  
\end{equation}

The maturity levels presented by NATO in SAS-065 report show the ability of an entity or a group of entities to change the C2 Approach under circumstances changes \cite{nato01}. C2 Agility will guarantee the right position in the C2 Approach space for their mission.

It is a requirement to be aware of where the entities are in the C2 Approach Space. This is the first step to maneuver in this space to get new circumstances adaptation.

When the circumstances' requirements cause a significant location change in the endeavor space, there is a mapping to a different region and consequently we have a C2 Approach modification. As a result, the approach is the appropriate one to deal with the new circumstances.


\section {Software Product Line}

Sven et al. \cite{SPL10} defines a Software Product Line (SPL) as following:

\textit{Software product line is an approach that provides a form of mass customization by constructing individual solutions based on a portfolio of reusable software components. It introduces individualism into software production, but still retains the benefits of mass production in that whole domain and market segments can be served}

\figura[!h]{spl01}{SPL lifecycles (from original in \cite{SPL10})}{spl01}{width=1\textwidth}%

The SPL has two lifecycles as seen in Figure \ref{spl01}. The Domain Engineering defines the commonality and the variability of the product line based on the domain aspects, rules and requirements. It defines the common and exchangeable parts of the system, guarantying the creation of reusable artifacts with traceability.

Application engineering relates the software product to the reusable domain specific artifacts, exploring the capability to generate products from a common base of components and/or artifacts.


<<TO IMPROVE>>

The current studies do not explore the element of communication among DSPL and how this integration impacts on the final configuration and collaboration between the entities. Based on a multi-agent system, the communication and reasoning matter. These two complex elements are present when a set of agents works together to obtain a result.
As DSPL can be seen as an agent modelling and implementation, the communication aspect matters when we are analyzing the interaction among more than one DSPL.

The communication structure and protocol will define how the elements of a team exchange information about their configuration and collaboration. This dialogue follows a flow that defines how the entities are organized inside the team or among the teams.

Even not existing collaboration in tasks execution, this interaction contributes with the configuration because it guarantees to configure the elements in a optimized way according to the tasks to be done.

Figure\ref{dspl} shows the DSPL adaptation is realized by binding variation points at runtime repeatedly. In this case, each valid configuration presented on the valid configuration space is linked with an arrow that indicates how the configuration changes are performed in case of context modification.

\figura[!h]{DSPL01}{DSPL reconfiguration with context changes}{dspl}{width=.5\textwidth}%




\section {Self-Adaptive Systems}

Laddaga et al. in \cite{SAS01} presented the following definition to self-adaptive system (SAS):

\textit{Self-adaptive software evaluates its own behavior and changes behav- ior when the evaluation indicates that it is not accomplishing what the software is intended to do, or when better functionality or performance is possible.}

This definition comes from self-management software \cite{SAS02}, where it is not necessary an administrator to manage the system. This aspect brings four properties inherent of these systems:
\begin{itemize}
    \item \textbf{self-configuration:} the system is able to configure itself according to policies and requirements from the environment;
    \item \textbf{self-optimization:} the system adjusts the adaptations to obtain the best possible results and improve its performance;
    \item \textbf{self-healing:} the system is capable to identify and to correct possible problems occurred;
    \item \textbf{self-protection:} the system is capable to protect itself against attacks and threats.
\end{itemize}

A self-adaptive system requires a structure to perform all logical operations to execute its adaptation. This structure can be external or internal if compared with the system core \cite{SAS03}. When the adaptation logic is part of the main system, it is classified as internal. It is external when the adaptation logic is a separated module or component. 

In the second case, the maintainability is higher and it is easier to reuse and exchange components. Otherwise, the internal adaptation is faster but it is more dependent of the main system. However, both kind of structure are the most important component of a SAS.

Cheng et al. \cite{SAS04} uses feedback loops to implement SAS. These loops are formed by states that collect information from the environment and/or system and plan an execution to adapt itself, if necessary, according to the new situation. A generic loop used by that work is the MAPE cycle, defined by the stages: \textit{Monitor}, \textit{Analyze}, \textit{Plan}, and \textit{Execute}.

\begin{itemize}
    \item \textbf{Monitor:} collect data from all managed elements;
    \item \textbf{Analyze:} process all data collected applying metrics and constraints;
    \item \textbf{Plan:} defines necessary changes to get the best results or to fix some issue identified in the previous phase;
    \item \textbf{Execute:} executes the plan developed to adapt the system.
\end{itemize}

Software Product Line (SPL) is an approach used to model a SAS \cite{SPL10}.


